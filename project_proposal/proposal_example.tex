\documentclass[PI]{ProjectProposal}
% Возможные опции: PI или BI

\title{Project proposal example}
\author{Ivan Ivanov Ivanovish}
\supervisor{position}{academic degree}{name surname}
\Group{PI-19-1}
\Year{2022}

% Ссылка на файл с описание библиографии
\bibliography{library.bib}

%%%%%%%%%%%%%%%%%%%%%%%%%%%%%%%%
%%% ТЕКСТ РАБОТЫ %%%%%%%%%%%%%%%
\begin{document}

\maketitle
Аннотация представляет собой краткое изложение работы с указанием:
\begin{itemize}
  \item цели исследования;
  \item методов исследования и выборки
  \item предполагаемых результатов проведенного исследования
  \item структуры работы.
\end{itemize}
Слово «Abstract» не пишется. Аннотация состоит из одного абзаца и располагается на первой странице непосредственно перед основным текстом, отделяется от него двумя пробелами и по объему не должна превышать 10\% от количества слов в основных информативных частях работы (введение, основная часть, заключение).

Аннотация (Abstract) - TL;DR summarize each of parts below, should reflect the whole paper. Should be about 10\% or less of project proposal aka 200-250 words.
\chapter*{Introduction}
Подзаголовки части Introduction (Background, Problem Statement, Delimitations of the Study, Professional Significance, Definitions of Key Terms) пишутся в строку, выделяются жирным шрифтом и отделяются от основного текста точкой.

В подразделах Introduction обосновывается актуальность выбранной темы (Background), определяются цели и задачи исследования (Problem Statement), определяется рассматриваемый круг вопросов (Delimitations of the Study), раскрывается, при возможности, практическая значимость проводимого исследования и\textbackslash или научная новизна решаемых задач (Professional Significance), при необходимости даются определения ключевых терминов (Definitions of Key Terms) с обязательным указанием источников. Рекомендуемый объем - 500 слов \cite{HSEDocuments}.
\chapter*{Literature review}
- анализ литературных источников,
Анализ литературы раскрывает состояние исследуемой проблемы в определенной области научных знаний с обоснованием направления исследования. Текст должен носить аналитический характер. внутритекстовые сноски оформляются в соответствии с требованиями АРА (фамилия автора, год) (http://www.thewritedirection.net/apaguide.net/apaguide.pdf). Рекомендуемый объем – 1300 слов.
\chapter*{Methods}
- обоснование выбора методов
Раздел Методы включает в себя краткое описание методов исследования с обоснованием их выбора. (Для образовательных программ Бизнес информатика и Программная инженерия – обзор краткое описание методов проектирования/моделирования/реализации задач с обоснованием их выбора) Рекомендуемый объем – 300 слов.


\chapter*{Results Anticipated}
- описание предполагаемых (или достигнутых на момент сдачи проекта) результатов.
Раздел Предполагаемые результаты содержит описание (предполагаемых) результатов исследования, формулировка результатов должна коррелировать с поставленными задачами и выбранными методами. Рекомендуемый объем – 200 слов

\chapter*{Conclusion}
Заключение представляет собой последовательное изложение полученных итогов и их соотношение с целью и задачами и практической значимостью, поставленными и сформулированными во введении.

\section{References}
Список используемой литературы (References) представляет собой список использованных в работе источников. В него могут входить статьи, монографии, книги, справочная литература и пр., а также информация, размещенная на академических электронных ресурсах.


Список источников приводится в алфавитном порядке по фамилиям авторов и оформляется по правилам академического стиля АРА (cм. ниже) и формируется исходя из рекомендаций научного руководителя. Минимальное количество источников, используемых в работе, - 10. Из этих 10 источников минимум 8 должны быть англоязычными. При необходимости использовать русскоязычные источники, они оформляются на русском языке в том же формате, что и англоязычные источники, и приводятся в конце списка. На все источники, указанные в списке, должны иметься ссылки в тексте работы.

Источники перечисляются в алфавитном порядке, с использованием так называемого <<висячего отступа>> (противоположность традиционной красной строки): первая строка данных источника начинается слева, без отступа, а все последующие строки – с отступом.

Названия книг и журналов выделяются курсивом. В заголовках книг и названиях журналов все слова, кроме предлогов, артиклей и союзов пишутся с заглавной буквы.  В заголовках статьей с заглавной буквы пишется только первое слово. Также заглавные буквы используются везде для имен собственных и аббревиатур.

При включении в список источников более чем одной работы одного автора, работы приводятся в порядке года их издания.

При отсутствии автора, работа приводится в списке по алфавиту в соответствии с названием, а в тексте при цитировании указывается сокращенное название работы.

Пример оформления цитирования в latex:
\begin{itemize}
  \item \verb\cite{ConferencePaperExample}\ -- \cite{ConferencePaperExample}
  \item \verb\citeauthor{ConferencePaperExample}\ -- \citeauthor{ConferencePaperExample}
  \item \verb\textcite{ConferencePaperExample}\ -- \textcite{ConferencePaperExample}
  \item \verb\autocite{ConferencePaperExample}\ -- \autocite{ConferencePaperExample}
  \item \verb\autocite*{ConferencePaperExample}\ -- \autocite*{ConferencePaperExample}
  \item \verb\autocite{ConferencePaperExample}\ -- \autocite{ConferencePaperExample}
\end{itemize}

\nocite{*} % для примера библиографии, так лучше не делать
\putbibliography

\end{document}
