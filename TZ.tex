% https://www.swrit.ru/doc/gost34/34.602-2020.pdf
% Сделано по гост 34.602-2020
{ % чтобы в главном тексте не сбрасывались заголовки
%%%%%%%%%%%%%%%%%%%%% переопределение секций чтобы было 1.1. вместо А.1.1.
\renewcommand*{\thesection}{\arabic{section}}
\titleformat{\section}{\large\bfseries}{\thesection.}{4pt}{}

\renewcommand*{\thesubsection}{\arabic{section}.\arabic{subsection}}
\titleformat{\subsection}{\large\bfseries}{\thesubsection.}{4pt}{}
%%%%%%%%%%%%%%%%%%%%%%%%%%5
{ % для титульного листа
\chapter*{ПРИЛОЖЕНИЕ А Техническое задание на разрабатываемую систему}
\stepcounter{chapter}
\thispagestyle{empty}
\centering

\begin{flushright}
  \MakeUppercase{Утверждено}

  A.B.00001-01 ТЗ 01
\end{flushright}

\vfill

\textbf{\MakeUppercase{Заголовок}}

\textbf{Техническое задание}

\textbf{\textit{Лист утверждения}}

\vbox{
  \parbox{6cm}{
    \begin{sideways}
      \setlength\arrayrulewidth{2pt}
      \begin{tabular}{|c|c|c|c|c|}
        \hline
        Инв. № подл. & Подпись и дата & Взам. инв. № & Инв. № дубл. & Подпись и дата \\
        \hline
                     &&&&\\
        \hline
      \end{tabular}
    \end{sideways}
  }
  \hfill
  \parbox{9.4cm}{
    \begin{flushright}
      \begin{minipage}[t]{0.4\textwidth}
        Руководитель разработки

        \vspace{3mm}
        \makebox[6cm][r]{\hrulefill~Иванов~И.И.}
        \makebox[6cm][r]{<<\rule{7mm}{0.4pt}>>\hrulefill~\the\year}
      \end{minipage}
    \end{flushright}
    \vspace{10mm}
    \begin{flushright}
      \begin{minipage}[t]{0.4\textwidth}
        Исполнитель

        \vspace{3mm}
        \makebox[6cm][r]{\hrulefill~Петров~П.П.}
        \makebox[6cm][r]{<<\rule{7mm}{0.4pt}>>\hrulefill~\the\year}
      \end{minipage}
    \end{flushright}
  }
}
\newpage
}
\section{Общие сведения}
В разделе «Общие сведения» указывают следующее:
\begin{itemize}
    \item полное наименование АС и ее условное обозначение
    \item шифр темы (при наличии);
    \item наименование организации — заказчика АС, наименование организации-разработчика (при наличии сведений о ней);
    \item перечень документов, на основании которых создается АС, кем и когда утверждены эти документы
    \item плановые сроки начала и окончания работ по созданию АС
    \item общие сведения об источниках и порядке финансирования работ
\end{itemize}

П р и м е ч а н и е — К документам, на основании которых или в соответствии с которыми создается АС, могут относиться, например, следующие:\\
- договорные документы на создание АС;\\
- нормативно-правовые и нормативно-технические документы, регламентирующие создание АС;\\
- техническое задание на создание ранее разрабатывавшейся АС.
\section{Цели и назначение создания автоматизированной системы}

\subsection{Цели создания АС}
В подразделе «Цели создания АС» приводят наименования и требуемые значения технических, технологических, производственно-экономических или других показателей объекта автоматизации, которые должны быть достигнуты в результате создания АС, и указывают критерии оценки достижения целей создания АС

\section{Характеристика объекта автоматизации}
В разделе «Характеристика объекта автоматизации» приводят следующую информацию:\\
- основные сведения об объекте автоматизации или ссылки на документы, содержащие такие сведения;\\
- сведения об условиях эксплуатации объекта автоматизации и характеристиках окружающей среды.\\
П р и м е ч а н и е — В разделе приводят основные сведения об объекте автоматизации, позволяющие одно значно его идентифицировать и сформировать правильное представление о масштабах разработки.

\section{Требования к автоматизированной системе}
Состав требований к АС, включаемых в данный раздел ТЗ на АС, устанавливают в зависимости
от вида, назначения, специфических особенностей и условий функционирования конкретной автома
тизированной системы. В каждом подразделе приводят ссылки на действующие НТД, определяющие
требования к автоматизированным системам соответствующего вида.
\subsection{Требования к структуре АС в целом}
В подразделе «Требования к структуре АС в целом» указывают следующее:\\
- перечень подсистем (при их наличии), их назначение и основные характеристики. Дополнительно могут быть приведены требования к числу уровней иерархии и степени централизации АС;\\
- требования к способам и средствам обеспечения информационного взаимодействия компонентов АС;\\
- требования к характеристикам взаимосвязей создаваемой АС со смежными АС, требования к интероперабельности, требования к ее совместимости, в том числе указания о способах обмена информацией;\\
- требования к режимам функционирования АС;\\
- требования по диагностированию АС;\\
- перспективы развития, модернизации АС.\\
\subsection{Требования к функциям (задачам), выполняемым АС}
В подразделе «Требования к функциям (задачам), выполняемым АС», приводят перечень функций (задач), подлежащих автоматизации для АС в целом или для каждой подсистемы (при их наличии). В перечень включаются в том числе функции  (задачи), обеспечивающие взаимодействие частей АС.
Для каждой функции (задачи) должен быть указан результат ее выполнения и, при необходимости, приведены основные арактеристики результата. При необходимости дополнительно могут быть указаны следующие данные:\\
- временной регламент реализации каждой функции (задачи);\\
- требования к реализации каждой функции (задачи), к форме представления выходной инфор мации, характеристики необходимой точности и времени выполнения, требования одновременности выполнения группы функций, достоверности выдачи результатов;\\
- перечень и критерии отказов для каждой функции, по которой задаются требования по надежности.\\
\subsection{Требования к видам обеспечения АС}
В подразделе «Требования к видам обеспечения АС» приводят требования к математиче
скому, информационному, лингвистическому, программному, техническому, метрологическому, органи
зационному, методическому и другим видам обеспечения АС.

Для математического обеспечения АС приводят требования к составу, области примене
ния (ограничениям) и способам использования в АС математических методов и моделей, типовых алго
ритмов и алгоритмов, подлежащих разработке.

Для информационного обеспечения АС приводят следующие требования:\\
- к составу, структуре и способам организации данных в АС;\\
- к информационному обмену между компонентами АС и со смежными АС;\\
- к информационной совместимости со смежными АС;\\
- по использованию действующих и по разработке новых классификаторов, справочников, форм документов;\\
- по применению систем управления базами данных;\\
- к представлению данных в АС;\\
- к контролю, хранению, обновлению и восстановлению данных.\\

Для лингвистического обеспечения АС приводят следующие требования:\\
- к языкам, используемым в АС, и возможности расширения набора языков (при необходимости);\\
- к способам организации диалога; \\
- к разработке и использованию словарей, тезаурусов;\\
- к описанию синтаксиса формализованного языка.\\

Для программного обеспечения АС приводят следующую информацию:\\
- требования к составу и видам программного обеспечения;\\
- требования к выбору используемого программного обеспечения;\\
- требования к разрабатываемому программному обеспечению;\\
- перечень допустимых покупных программных средств (при наличии).\\

Для технического обеспечения АС приводят следующие требования:\\
- к видам технических средств, в том числе к видам комплексов технических средств, программно-технических комплексов и других комплектующих изделий, допустимых к использованию в АС;\\
- к функциональным, конструктивным и эксплуатационным характеристикам средств технического обеспечения АС.\\

В требованиях к метрологическому обеспечению АС приводят следующую информацию:\\
- количественные значения показателей метрологического обеспечения;\\
- требования к методам (методикам) измерений и измерительного контроля параметров и их характеристик;\\
- требования к средствам измерений и измерительного контроля;\\
- требования к метрологическому обеспечению испытаний АС;\\
- требования к программе метрологического обеспечения АС;\\
- требования к метрологической совместимости технических средств АС;\\
- требования проведения метрологической экспертизы технической документации (при необходимости).\\

Для организационного обеспечения АС приводят следующие требования:\\
- к структуре и функциям подразделений, участвующих в функционировании АС или обеспечивающих эксплуатацию;\\
- к организации функционирования АС и порядку взаимодействия персонала и пользователей АС;\\
- к организации функционирования АС при сбоях, отказах и авариях;\\
- к порядку обеспечения нормативными документами, необходимыми для разработки АС.\\

Для методического обеспечения АС приводят следующую информацию:\\
- перечень применяемых при разработке и функционировании АС нормативно-технических документов (стандартов, нормативов, методик, профилей и т. п.);\\
- порядок и правила обеспечения разработчиков АС нормативно-технической документацией.\\
\subsection{Общие технические требования к АС}
В подразделе «Общие технические требования к АС» указывают следующее:\\
- требования к численности и квалификации персонала и пользователей АС;\\
- требования к показателям назначения;\\
- требования к надежности;\\
- требования по безопасности;\\
- требования к эргономике и технической эстетике;\\
- требования к транспортабельности для подвижных АС;\\
- требования к эксплуатации, техническому обслуживанию, ремонту и хранению компонентов АС;\\
- требования к защите информации от несанкционированного доступа;\\
- требования по сохранности информации при авариях;\\
- требования к защите от влияния внешних воздействий;\\
- требования к патентной чистоте и патентоспособности;\\
- требования по стандартизации и унификации;\\
- дополнительные требования.\\

В требованиях к численности и квалификации персонала и пользователей АС приводят следующее:
- требования к численности персонала и пользователей АС;\\
- требования к квалификации персонала и пользователей АС, порядку их подготовки и контроля знаний и навыков;\\
- требуемый режим работы персонала и пользователей АС.\\

В требованиях к показателям назначения АС приводят значения параметров, характеризу
ющих степень соответствия АС ее назначению (при их наличии).

В требования к надежности включают:\\
- состав и количественные значения показателей надежности для АС в целом или ее подсистем (составных частей);\\
- перечень аварийных ситуаций, по которым должны быть регламентированы требования к надежности, и значения соответствующих показателей; \\
- требования к надежности технических средств и программного обеспечения;\\
- требования к методам оценки и контроля показателей надежности на разных стадиях создания АС в соответствии с действующими нормативно-техническими документами.\\

В требования по безопасности включают требования по обеспечению безопасности при монтаже, наладке, эксплуатации, обслуживании и ремонте технических средств АС (защита от воздействий электрического тока, электромагнитных полей и т. п.), по допустимым уровням вибрационных и шумовых нагрузок, а также по обеспечению экологической безопасности.

В требования к эргономике и технической эстетике включают следующие требования:\\
- эргономические требования к организации и средствам деятельности персонала и пользователей АС, в том числе к средствам отображения информации и организации рабочего места;\\
- требования к технической эстетике, определяющие композиционную целостность, информационную выразительность, рациональность формы и культуру производственного исполнения создаваемого изделия, в том числе реализации человеко-машинного интерфейса.\\

В требования к транспортабельности для подвижных АС включают конструктивные требования, обеспечивающие транспортабельность технических средств АС, а также требования к транспортным средствам, включая условия транспортирования, возможность перевозки в готовом к функционированию состоянии, необходимость защиты элементов АС от внешних воздействующих факторов при транспортировании, а также требования безопасности перевозки.

В требования к эксплуатации, техническому обслуживанию, ремонту и хранению компонентов АС включают:\\
- условия и регламент (режим) эксплуатации, которые должны обеспечивать использование технических средств (ТС) и программно-технических средств (ПТС) АС с заданными показателями;\\
- требования к видам, периодичности и объему технического обслуживания, контролю технического состояния и ремонта или допустимость работы без обслуживания;\\
- предварительные требования к допустимым площадям для размещения персонала и технических средств АС, к параметрам сетей энергоснабжения, вентиляции, охлаждения и т. п.;\\
- требования к составу, размещению и условиям хранения комплекта запасных частей, инструментов и принадлежностей, а также к нормам расхода запасных частей;\\
- требования к регламенту обслуживания.\\

В требования к защите информации от несанкционированного доступа включают требования, установленные в НТД, действующей в отрасли (ведомстве) заказчика.

В требованиях по сохранности информации приводят перечень событий: аварий, отказов технических средств (в том числе — потеря питания) и т. п., при которых должна быть обеспечена сохранность информации в АС.

В требованиях к защите от внешних воздействий приводят:\\
- требования к радиоэлектронной защите средств АС;\\
- требования по стойкости, устойчивости и прочности к внешним воздействиям (среде применения)\\

В требованиях к патентной чистоте и патентоспособности указывают требования по патентной чистоте и патентоспособности АС и ее частей, включая требования по проведению патентных исследований.

В требования к стандартизации и унификации включают показатели, устанавливающие следующее:\\
- требуемую степень использования стандартных, унифицированных методов реализации функций (задач) АС, поставляемых программных средств, типовых математических методов и моделей, типовых проектных решений, унифицированных форм документов, общероссийских классификаторов и классификаторов других категорий в соответствии с областью их применения;\\
- требования к использованию типовых автоматизированных рабочих мест, компонентов и комплексов\\

В дополнительные требования включают:\\
- требования к оснащению АС учебно-тренировочными средствами и документацией на них;\\
- требования к сервисной аппаратуре, стендам для проверки элементов АС;\\
- требования к АС, связанные с особыми условиями эксплуатации;\\
- специальные требования по усмотрению разработчика или заказчика АС.\\

\section{Состав и содержание работ по созданию автоматизированной системы}
Раздел «Состав и содержание работ по созданию автоматизированной системы» должен содержать перечень этапов работ по созданию АС и сроки их выполнения.

\section{Порядок разработки автоматизированной системы}
В разделе «Порядок разработки автоматизированной системы» приводят следующее:\\
- порядок организации разработки АС;\\
- перечень документов и исходных данных для разработки АС;\\
- перечень документов, предъявляемых по окончании соответствующих этапов работ;\\
- порядок проведения экспертизы технической документации;\\
- перечень макетов (при необходимости), порядок их разработки, изготовления, испытаний, необходимость разработки на них документации, программы и методик испытаний;\\
- порядок разработки, согласования и утверждения плана совместных работ по разработке АС;\\
- порядок разработки, согласования и утверждения программы работ по стандартизации;\\
- требования к гарантийным обязательствам разработчика;\\
- порядок проведения технико-экономической оценки разработки АС;\\
- порядок разработки, согласования и утверждения программы метрологического обеспечения, программы обеспечения надежности, программы эргономического обеспечения.\\

\section{Порядок контроля и приемки автоматизированной системы}
В разделе «Порядок контроля и приемки автоматизированной системы» указывают следую
щую информацию:
- виды, состав и методы испытаний АС и ее составных частей;
- общие требования к приемке работ, порядок согласования и утверждения приемочной докумен
тации;
- статус приемочной комиссии (государственная, межведомственная, ведомственная и др.).

П р и м е ч а н и е — Порядок согласования и утверждения приемочной документации, а также статус при
емочной комиссии указываются при необходимости.

\section{Требования к составу и содержанию работ по подготовке объекта автоматизации
к вводу автоматизированной системы в действие}
В разделе «Требования к составу и содержанию работ по подготовке объекта автоматизации
к вводу автоматизированной системы в действие» приводят перечень мероприятий, которые необходи
мо осуществить при подготовке объекта автоматизации к вводу АС в действие.
В перечень мероприятий включают следующее:
- создание условий функционирования объекта автоматизации, при которых гарантируется соот
ветствие создаваемой АС требованиям, содержащимся в ТЗ на АС;
- проведение необходимых организационно-штатных мероприятий;
- порядок обучения персонала и пользователей АС.

\section{Требования к документированию}
В разделе «Требования к документированию» приводят следующую информацию:
- перечень подлежащих разработке документов;
- вид представления и количество документов;
- требования по использованию ЕСКД и ЕСПД при разработке документов.
При отсутствии государственных стандартов, определяющих требования к документированию
элементов АС, дополнительно включают требования к составу и содержанию таких документов.

\section{Источники разработки}
В разделе «Источники разработки» должны быть перечислены документы и информацион
ные материалы (технико-экономическое обоснование, отчеты о законченных научно-исследователь
ских работах, информационные материалы на отечественные, зарубежные системы-аналоги и др.), на
основании которых разрабатывалось ТЗ и которые должны быть использованы при создании АС.
}
